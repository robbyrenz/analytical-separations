%----------------------------------------------------------------------------------------
%	PACKAGES AND DOCUMENT CONFIGURATIONS
%----------------------------------------------------------------------------------------

\documentclass[a4paper, 12pt]{article}
\usepackage[version=4]{mhchem} % Package for chemical equation typesetting
\usepackage{siunitx} % Provides the \SI{}{} and \si{} command for typesetting SI units
\usepackage{amsmath} % Required for some math elements 
% \usepackage{times} % Uncomment to use the Times New Roman font
\usepackage[backend=biber, bibencoding=utf-8, style=chem-acs, citestyle=chem-acs, sorting=none]{biblatex}
\setlength{\parskip}{1em}
\addbibresource{references.bib}
\usepackage{geometry}
\geometry{margin=1in}
\usepackage{textcomp}

%----------------------------------------------------------------------------------------
%	DOCUMENT INFORMATION
%----------------------------------------------------------------------------------------

\title{An Introduction to Supercritical Fluid Chromatography and its Applications \\ CHEM-4303 \\ Analytical Separations} % Title

\author{Robby \textsc{Renz}} % Author name

\date{\today} % Date for the report

\begin{document}

\maketitle % Insert the title, author and date

%----------------------------------------------------------------------------------------
%	SECTION 1
%----------------------------------------------------------------------------------------

\section{Abstract}
Supercritical fluid chromatography (SFC) is a type of separation technique that uses a supercritical fluid as its choice of solvent \cite{harris}. The mobile phase commonly used is \ce{CO2}, a supercritical fluid at a certain pressure but a weak solvent, so because of this fact, it is mixed with some other organic solvent so as to increase its solubility in a lot of other compounds \cite{harris}. In many regards, it is quite similar to high performance liquid chromatography (HPLC) in terms of its equipment and software \cite{taylor_supercritical_2010}. However, the pump system in a SFC instrument must be equipped with a chilled head in order to stabilize \ce{CO2} in its liquid state \cite{taylor_supercritical_2010}. But it does have numerous advantages to the HPLC such as a much lower operating cost and solvent consumption \cite{taylor_supercritical_2010}. Due to this latter fact, this separation technique is considered environmentally friendly \cite{harris}. In addition, it has a faster flow rate as the viscosity of the mobile phase is extremely low \cite{harris}. This presentation will go into depth as to what exaclty a supercritical fluid is, how does a SFC function in terms of its equipment, and its applications in the separation of chiral compounds.


%----------------------------------------------------------------------------------------
%	SECTION 2
%----------------------------------------------------------------------------------------

\section{References}
\printbibliography

%----------------------------------------------------------------------------------------

\end{document}

