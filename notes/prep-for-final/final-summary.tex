\documentclass[a4paper, 12pt]{article}
\usepackage[version=4]{mhchem} % Package for chemical equation typesetting
\usepackage{siunitx} % Provides the \SI{}{} and \si{} command for typesetting SI units
\usepackage{amsmath} % Required for some math elements 
% \usepackage{times} % Uncomment to use the Times New Roman font
\usepackage[backend=biber, bibencoding=utf8, style=chem-acs, citestyle=chem-acs, sorting=none]{biblatex}
\setlength{\parskip}{1em}
% \addbibresource{references.bib}
\author{Robby \textsc{Renz}}
\title{A Summary of Analytical Separations}

\begin{document}
\maketitle
\newpage

\tableofcontents
\listoffigures
\listoftables
\newpage

\section{High Performance Liquid Chromatography}

\subsection{Introduction}
\begin{itemize}
	\item HPLC stands for "High Performance Liquid Chromatography" or "High Pressure Liquid Chromatography"
	\item \textbf{Advantages}
	\begin{itemize}
		\item Analysis of thermally unstable compounds
		\item Analysis of nonvolatile compounds
	\end{itemize}
	\item \textbf{Major requirement of LC}
	\begin{itemize}
		\item Solute solubility in mobile phase
		\item This is in contrast to GC which require solute volatility
	\end{itemize}
\end{itemize}


\subsection{Scope of HPLC}
\begin{description}
	\item[Adsorption chromatography (LSC)]
	\item[Ion chromatography (IC)]
	\item[Size-exclusion chromatography (SEC)]
	\item[Partition chromatography] separation of analytes by partitioning, most commonly to a stationary phase bonded to a solid support \\ Note that this replaces liquid-liquid chromatography with its problems of stripping of stationary phase
	\item[Hydrophobic interaction chromatography (HIC)] for separation of proteins without denaturation
	\item[Hydrophilic interaction chromatography (HILIC)] for separation of very polar analytes
	\item[Chiral chromatography]
	\item[Affinity chromatography]
\end{description}


\subsection{Column Efficiency in HPLC}
\begin{itemize}
	\item Recall the van Deemter equation for GLC (commonly called GC) for packed columns:
	\begin{equation}
		H = A + \frac{B}{u} + Cu
	\end{equation}
	\item For longitudinal diffusion in LC, as the $D_l \approx 10^{-5}D_g$, the peak broadening due to longitudinal diffusion in mobile phase (liquid) phase in LC is \textit{negligible}.
	\item i.e., $\frac{B}{u} = \frac{2\gamma{}D_m}{u}$
\end{itemize}


\subsection{Pumps}

\begin{description}
	\item[High Pressure Pumps] required to force liquid through a densely packed column at constant flow rate, so that $t_r$ is reproducible.
\end{description} 

Requirements of a pump:

\begin{itemize}
	\item output pressure all the way to 6000 psi, without any leaks; for UHPLC, the pressure is typically $\approx$ \SI{18000}{psi}
	\item have variable flow rates, from \SIrange{0.1}{10}{mL/min}; for UHPLC, the pressure can go all the way down to \SI{0.01}{mL/min}
	\item have a pulse-free output
	\item have a reproducible flow rate (ours to \SI{+-0.3}{\%})
\end{itemize}

\subsubsection{Reciprocating Pumps}
These are the most common today, replacing pneumatic and displacement (syringe) pumps.

A single piston reciprocating pump:

\begin{itemize}
	\item an eccentric (off-center) can drives a piston back and forth
	\item its motion is synchronized with operation of check valves which control direction of flow
	\item Fill stroke: piston cavity fills, drawing solvent through the inlet check valve via suction; the outlet valve is closed
	\item delivery stroke: mobile phase forced through column as inlet check valve closes and the outlet valve opens. Pulsed flow produces baseline noise that must be damped
	\item electronic pulse compensation: speeds up piston during refill cycle, reducing the time the piston is not delivering solvent ($\approx$ \SI{200}{ms}); this reduces the noisy baseline
	\item note that the pulse flow pulsates, but the time-averaged flow rate is constant
	\item the regular pressure surge spikes in the baseline may still be seen at extreme sensitive detector
	\item an elliptical cam minimize pulsations
\end{itemize}

\subsubsection{Dual Piston Reciprocating Pumps}

\begin{itemize}
	\item essentially 2 single-piston pumps driven by the same motor
	\item 2 pistons are \ang{180} out of phase, i.e., they are synchronized to provide a continuous flow of mobile phase
	\item when one is in the fill stroke, the other is on deliver stroke
\end{itemize}

\subsubsection{HPLC solvents (Mobile Phase)}

Solvents need to be HPLC or of similar high-grade for a stable baseline and to extend the life of the column, 'cause those things are expensive!

\begin{description}
	\item[Filtration] to remove particulate matter
	\item[Degassing] to remove dissolved gases, especially \ce{N2} and \ce{O2}, by vacuum or \ce{He} sparging 
\end{description}

\subsubsection{Pressure Drop Across Column (Backpressure)}

\begin{itemize}
	\item Pressure developed when the liquid mobile phase is pumped through the (packed) HPLC column: 

*enter the equation for backpressure*

	\item therefore, for much larger backpressure occurs with smaller particles, more tightly packed column and a more viscous mobile phase
	\item there is a tradeoff between better performance and higher pressure
\end{itemize}


\subsection{Elution}

\begin{itemize}
	\item Unlike GC, the mobile phase important in LC separations
	\item Analyte molecules and solvent molecules can compete with each other for binding sites on the stationary phase 
	\item The greater the mobile phase eluent strength, the more easily solvent molecules displace analyte molecules from the stationary phase
	\begin{itemize}
		\item A stronger mobile phase results in lower k and $t_r$
	\end{itemize}
	\item Layers of solvent at the interface affect separations
	\begin{itemize}
		\item e.g., for reversed phase separations, methanol forms a monolayer on a C18 surface, while acetonitrile forms a pool 1.3 nm deep into which analytes can dissolve with different binding energies.
		\item Net result: acetonitrile is a stronger solvent than MeOH
	\end{itemize}
	\item The solvent choice also influences peak shape
\end{itemize}

\subsubsection{Elution Techniques}

\begin{itemize}
	\item Two major types of elution techniques in HPLC:
	\begin{itemize}
		\item Isocratic elution (one solvent)
		\item elution with a mobile phase with an unchanging composition over course of separation
		\begin{itemize}
			\item e.g., 80/20 \ce{CH3OH}:\ce{H2O}, 50/50 \ce{CH3CN}:\ce{H2O}
			\item Analogous to isothermal GC
		\end{itemize}
		\item Gradient elution
		\item technique in which mobile phase composition gradually changed over course of separation
		\begin{itemize}
			\item Analogous to programmed temperature GC
			\item Gradient can be ionic strength, pH, and/or organic modifier (e.g., \ce{CH3OH} in water)
			\item Used to improve resolution and sensitivity
			\item Done with proportioning valve in pump
		\end{itemize}
	\end{itemize}
\end{itemize}

\subsubsection{Factors to Consider in Gradient Elution}

\begin{itemize}
	\item Be sure that solvents are miscible in proportions used AND that solutes are soluble in all solvent compositions used during gradient elution
	\item Column equilibration (i.e., solvent and stationary phase) is required prior to each injection, otherwise $k$ changes and $t_r$ is not reproducible
	\begin{itemize}
		\item Requires 10-20 column volumes, therefore expensive
	\end{itemize}
		\item Organics (usually phthalates) present in the water used as the solvent tend to interfere
	\begin{itemize}
		\item Concentrate at the head of the non-polar column, then elute as the organic modifier (e.g., \ce{CH3OH}) percentage increases
		\item Limits sensitivity
	\end{itemize}
	\item Gradient elution less reproducible, more expensive than isocratic!
\end{itemize}


\subsection{Injectors}

\begin{itemize}
	\item Although various techniques (e.g., septum/syringe, stop-flow injection) have been used, most common device is a sampling valve
	\item Rotary sampling valve: injects sample from a sample loop using flow switching without depressurizing system
	\begin{itemize}
		\item LOAD: sample loop loaded with syringe at atmospheric pressure
		\begin{itemize}
			\item Mobile phase by-passes loop
		\end{itemize}
		\item INJECT:  turning rotor changes flow lines, bringing loop into line, and sample swept into column
		\begin{itemize}
			\item No flow interruption, only a slight pressure surge (turn rotor quickly to prevent large pressure pulse)
			\item For best reproducibility in quantitative work (\SI{+-0.2}{\%}) use loop overload (precalibrated, in sizes from \SIrange{0.1}{1000}{\mu{}L})
		\end{itemize}
	\end{itemize}
\end{itemize}


\subsection{Columns}


\subsection{Detectors}


\subsection{Types of Chromatography in HPLC}


\section{GC-MS}


\section{Important Tips}

\begin{description}
	\item[Isocratic elution] one solvent, or constant solvent mixture.
	\item[Gradient elution] continuous change of solvent composition to increase eluent strength. \\ Gradient elution in HPLC is analoguous to temperature programming in gas chromatography. \\ Increased eluent strength is required to elute more strongly retained solutes.
	\item[General elution problem] for a complex mixture, isocratic conditions can often be found to produce adequate separation of early-eluting peaks or late-eluting peaks, but not both. This problem drives us to use gradient elution.
	\item[Note:] Elution strength decreases as the solvent becomes more polar, correct???
	\item[Separation factor, $\alpha$] Also called relative retention; for two components, 1 and 2, it is the ratio of their adjusted retention times. \\ The greater the relative retention, the greater the separation between two components. \\ Relative retention is fairly independent of flow rate and can therefore be used to to help identify peaks when the flow rate changes. (show equation?)
\end{description}




\end{document}
