\documentclass[a4paper, 12pt]{article}
\usepackage[version=4]{mhchem} % Package for chemical equation typesetting
\usepackage{siunitx} % Provides the \SI{}{} and \si{} command for typesetting SI units
\usepackage{amsmath} % Required for some math elements 
% \usepackage{times} % Uncomment to use the Times New Roman font
\usepackage[backend=biber, bibencoding=utf8, style=chem-acs, citestyle=chem-acs, sorting=none]{biblatex}
\setlength{\parskip}{1em}
% \addbibresource{references.bib}
\author{Robby \textsc{Renz}}
\title{A Summary of Analytical Separations}

\begin{document}
\maketitle
\newpage

\tableofcontents
\listoffigures
\listoftables
\newpage

\section{High Performance Liquid Chromatography}

\subsection{Introduction}
\begin{itemize}
	\item HPLC stands for "High Performance Liquid Chromatography" or "High Pressure Liquid Chromatography"
	\item \textbf{Advantages}
	\begin{itemize}
		\item Analysis of thermally unstable compounds
		\item Analysis of nonvolatile compounds
	\end{itemize}
	\item \textbf{Major requirement of LC}
	\begin{itemize}
		\item Solute solubility in mobile phase
		\item This is in contrast to GC which require solute volatility
	\end{itemize}
\end{itemize}


\subsection{Scope of HPLC}
\begin{description}
	\item[Adsorption chromatography (LSC)]
	\item[Ion chromatography (IC)]
	\item[Size-exclusion chromatography (SEC)]
	\item[Partition chromatography] separation of analytes by partitioning, most commonly to a stationary phase bonded to a solid support \\ Note that this replaces liquid-liquid chromatography with its problems of stripping of stationary phase
	\item[Hydrophobic interaction chromatography (HIC)] for separation of proteins without denaturation
	\item[Hydrophilic interaction chromatography (HILIC)] for separation of very polar analytes
	\item[Chiral chromatography]
	\item[Affinity chromatography]
\end{description}


\subsection{Column Efficiency in HPLC}
\begin{itemize}
	\item Recall the van Deemter equation for GLC (commonly called GC) for packed columns:
	\begin{equation}
		H = A + \frac{B}{u} + Cu
	\end{equation}
	\item For longitudinal diffusion in LC, as the $D_l \approx 10^{-5}D_g$, the peak broadening due to longitudinal diffusion in mobile phase (liquid) phase in LC is \textit{negligible}.
	\item i.e., $\frac{B}{u} = \frac{2\gamma{}D_m}{u}$
\end{itemize}


\subsection{Pumps}

\begin{description}
	\item[High Pressure Pumps] required to force liquid through a densely packed column at constant flow rate, so that $t_r$ is reproducible.
\end{description} 

Requirements of a pump:

\begin{itemize}
	\item output pressure all the way to 6000 psi, without any leaks; for UHPLC, the pressure is typically $\approx$ \SI{18000}{psi}
	\item have variable flow rates, from \SIrange{0.1}{10}{mL/min}; for UHPLC, the pressure can go all the way down to \SI{0.01}{mL/min}
	\item have a pulse-free output
	\item have a reproducible flow rate (ours to \SI{+-0.3}{\%})
\end{itemize}

\subsubsection{Reciprocating Pumps}
These are the most common today, replacing pneumatic and displacement (syringe) pumps.

A single piston reciprocating pump:

\begin{itemize}
	\item an eccentric (off-center) can drives a piston back and forth
	\item its motion is synchronized with operation of check valves which control direction of flow
	\item Fill stroke: piston cavity fills, drawing solvent through the inlet check valve via suction; the outlet valve is closed
	\item delivery stroke: mobile phase forced through column as inlet check valve closes and the outlet valve opens. Pulsed flow produces baseline noise that must be damped
	\item electronic pulse compensation: speeds up piston during refill cycle, reducing the time the piston is not delivering solvent ($\approx$ \SI{200}{ms}); this reduces the noisy baseline
	\item note that the pulse flow pulsates, but the time-averaged flow rate is constant
	\item the regular pressure surge spikes in the baseline may still be seen at extreme sensitive detector
	\item an elliptical cam minimize pulsations
\end{itemize}

\subsubsection{Dual Piston Reciprocating Pumps}

\begin{itemize}
	\item essentially 2 single-piston pumps driven by the same motor
	\item 2 pistons are \ang{180} out of phase, i.e., they are synchronized to provide a continuous flow of mobile phase
	\item when one is in the fill stroke, the other is on deliver stroke
\end{itemize}

\subsubsection{HPLC solvents (Mobile Phase)}

Solvents need to be HPLC or of similar high-grade for a stable baseline and to extend the life of the column, 'cause those things are expensive!

\begin{description}
	\item[Filtration] to remove particulate matter
	\item[Degassing] to remove dissolved gases, especially \ce{N2} and \ce{O2}, by vacuum or \ce{He} sparging 
\end{description}

\subsubsection{Pressure Drop Across Column (Backpressure)}

\begin{itemize}
	\item Pressure developed when the liquid mobile phase is pumped through the (packed) HPLC column: 

*enter the equation for backpressure*

	\item therefore, for much larger backpressure occurs with smaller particles, more tightly packed column and a more viscous mobile phase
	\item there is a tradeoff between better performance and higher pressure
\end{itemize}


\subsection{Elution}

\begin{itemize}
	\item Unlike GC, the mobile phase important in LC separations
	\item Analyte molecules and solvent molecules can compete with each other for binding sites on the stationary phase 
	\item The greater the mobile phase eluent strength, the more easily solvent molecules displace analyte molecules from the stationary phase
	\begin{itemize}
		\item A stronger mobile phase results in lower k and $t_r$
	\end{itemize}
	\item Layers of solvent at the interface affect separations
	\begin{itemize}
		\item e.g., for reversed phase separations, methanol forms a monolayer on a C18 surface, while acetonitrile forms a pool 1.3 nm deep into which analytes can dissolve with different binding energies.
		\item Net result: acetonitrile is a stronger solvent than MeOH
	\end{itemize}
	\item The solvent choice also influences peak shape
\end{itemize}

\subsubsection{Elution Techniques}

\begin{itemize}
	\item Two major types of elution techniques in HPLC:
	\begin{itemize}
		\item Isocratic elution (one solvent)
		\item elution with a mobile phase with an unchanging composition over course of separation
		\begin{itemize}
			\item e.g., 80/20 \ce{CH3OH}:\ce{H2O}, 50/50 \ce{CH3CN}:\ce{H2O}
			\item Analogous to isothermal GC
		\end{itemize}
		\item Gradient elution
		\item technique in which mobile phase composition gradually changed over course of separation
		\begin{itemize}
			\item Analogous to programmed temperature GC
			\item Gradient can be ionic strength, pH, and/or organic modifier (e.g., \ce{CH3OH} in water)
			\item Used to improve resolution and sensitivity
			\item Done with proportioning valve in pump
		\end{itemize}
	\end{itemize}
\end{itemize}

\subsubsection{Factors to Consider in Gradient Elution}

\begin{itemize}
	\item Be sure that solvents are miscible in proportions used AND that solutes are soluble in all solvent compositions used during gradient elution
	\item Column equilibration (i.e., solvent and stationary phase) is required prior to each injection, otherwise $k$ changes and $t_r$ is not reproducible
	\begin{itemize}
		\item Requires 10-20 column volumes, therefore expensive
	\end{itemize}
		\item Organics (usually phthalates) present in the water used as the solvent tend to interfere
	\begin{itemize}
		\item Concentrate at the head of the non-polar column, then elute as the organic modifier (e.g., \ce{CH3OH}) percentage increases
		\item Limits sensitivity
	\end{itemize}
	\item Gradient elution less reproducible, more expensive than isocratic!
\end{itemize}


\subsection{Injectors}

\begin{itemize}
	\item Although various techniques (e.g., septum/syringe, stop-flow injection) have been used, most common device is a sampling valve
	\item Rotary sampling valve: injects sample from a sample loop using flow switching without depressurizing system
	\begin{itemize}
		\item LOAD: sample loop loaded with syringe at atmospheric pressure
		\begin{itemize}
			\item Mobile phase by-passes loop
		\end{itemize}
		\item INJECT:  turning rotor changes flow lines, bringing loop into line, and sample swept into column
		\begin{itemize}
			\item No flow interruption, only a slight pressure surge (turn rotor quickly to prevent large pressure pulse)
			\item For best reproducibility in quantitative work (\SI{+-0.2}{\%}) use loop overload (precalibrated, in sizes from \SIrange{0.1}{1000}{\mu{}L})
		\end{itemize}
	\end{itemize}
\end{itemize}


\subsection{Columns}

Learning objectives:
\begin{itemize}
	\item Size (conventional vs. prep-scale vs. UHPLC)
	\item Particle packing size (prep-scale to conventional to UHPLC)
	\item Packing type (normal vs. reversed-phase, linking, etc.)
\end{itemize}

\subsubsection{General Types of Analytical HPLC Columns}

\begin{itemize}
	\item HPLC columns usually straight, stainless steel tubing containing packing of 3-10 $\mu{}$m $d_p$ (for UHPLC, its about 1.7 $\mu{}$m)
	\item Analytical columns: for quantitative analysis
	\begin{itemize}
		\item Conventional: 2.1-4.6 mm i.d., 10-30 cm long, 0.5-2 mL/min (1 mL/min most common)
		\item Fast HPLC: 2.1-4.6 mm i.d., 3-10 cm long, 2-4 mL/min
		\begin{itemize}
			\item Shorter, faster flow acceptable
			\item Very low extracolumn volumes required in instrument
		\end{itemize}
		\item UHPLC: 2.1 mm i.d., 2-5 cm long, $\approx$\SIrange{0.5}{1}{mL/min}
		\item Capillary HPLC: 0.15-0.32 mm i.d., 5-10 cm long, 0.01 mL/min
		\begin{itemize}
			\item Provides similar linear flow rates as conventional columns
		\end{itemize}
	\end{itemize}
\end{itemize}

\subsubsection{Guard and Preparative HPLC Columns}

\begin{itemize}
	\item Guard columns:
	\begin{itemize}
		\item Short column containing similar packing as analytical packing
		\item Larger $d_p$ to avoid increasing backpressure ($\Delta$P)
		\item Removes impurities from sample/solvent, protecting the expensive (\$500-\$2000+) analytical column
	\end{itemize}
	\item Preparative columns:
	\begin{itemize}
		\item Larger i.d. (e.g., 22.4 mm) and length (e.g., 25-100 cm) to increase sample capacity
		\item Used to isolate and purify chemicals e.g., pharmaceutical, biochemical applications
		\item Flow rates to \SI{20}{mL/min}
		\item Requires specialty pump capable of handling high flow
	\end{itemize}
\end{itemize}


\subsection{Detectors}

\begin{itemize}
	\item Detector selection based on analytes to be detected
	\item Want a device that can provide a signal and operate in flow-through mode
	\item A wide variety of HPLC detectors are available, with different applications, advantages, drawbacks, sensitivities
	\item Optical detectors dominate market
\end{itemize}

\subsubsection{HPLC Absorbance Detectors}

\begin{itemize}
	\item Measure changes in light absorbance as solutes swept through small ($<$10 $\mu{}$L) cell
	\item Thus, absorbance detectors measure a property of the analyte
	\item Characteristics:
	\begin{itemize}
		\item Selective (response depends on molecular structure, i.e., $\epsilon$($\lambda$) = molar absorptivity)
		\item Nondestructive, can collect effluent to isolate compounds manually or with an automated fraction collector
		\item Good limits of detection (LOD):  ng to as low as 1 pg when $\epsilon$ large (e.g., 10,000 to 20,000)
		\item Compatible with gradient elution
		\begin{itemize}
			\item But must not use solvents which don’t absorb at wavelength of interest, i.e., are UV-transparent
		\end{itemize}
	\end{itemize}
\end{itemize}

\subsubsection{Fixed Wavelength Detector (Filter Photometer)}

\begin{itemize}
	\item Uses low pressure Hg lamp, isolating intensive 254 nm emission line with filter
	\item Other Hg emission lines can be isolated with proper filters
	\begin{itemize}
		\item 250, 313, 365 nm for example. There are others.
	\end{itemize}
	\item Detection is restricted to compounds absorbing at 254 nm or other emission line used
	\item Quantification is based on Beer’s law: absorbance (A) is directly proportional to the extinction coefficient ($\epsilon$), path length ($l$), and concentration ($c$):  $A(\lambda) = \epsilon(\lambda)lc$
	\item Note that different compounds have different $\epsilon(\lambda)$, so for a given chromatogram, peak heights don’t reflect relative analyte concentrations
	\item Requires careful solute-by-solute calibration
\end{itemize}

\subsubsection{Variable Wavelength Detector (Spectrophotometer)}

\begin{itemize}
	\item Uses continuous or broad-band light source e.g., $D_2$ lamp 190-400 nm
	\item Monochromator (grating optics) required for the dispersion of radiation and selection of a narrow bandpass for quantitative work
	\item Allows flexibility in choice of $\lambda$, which can be changed over course of run
	\item Operates otherwise like single wavelength detector, in that only 1 wavelength is measured at a time
	\item Wavelength switching is possible with appropriate software
\end{itemize}

\subsubsection{Diode Array Detector (DAD)}

\begin{itemize}
	\item Uses an array of photodiodes (transducer which converts photon flow to electrical energy) to monitor many wavelengths simultaneously, rather than sequentially as with variable $\lambda$ detector
	\begin{itemize}
		\item Typical DAD has 328 diodes, each monitoring a 2 nm $\lambda$ range, with circuitry on a 2 mm $\times$ 18 mm chip
	\end{itemize}
	\item Obtains a complete spectrum in $<<$ 1 sec
	\item Overcomes slow $\lambda$ scanning, too slow for HPLC with moving solutes
	\item Obtains spectra in 3D with appropriate software i.e., obtain complete spectra at specified sampling rate
	\begin{itemize}
		\item A vs. $\lambda$:  absorption spectra of each component as it passes through detector
		\item A vs. t: chromatogram
	\end{itemize}
\end{itemize}

\subsubsection{Advantages of a Diode Array Detector (DAD)}

\begin{itemize}
	\item Increased sensitivity
	\begin{itemize}
		\item Allows for quantitative determination of each solute at its $\lambda{}_{max}$
	\end{itemize}
	\item More information is acquired as a full UV-vis absorption spectrum is obtained
	\begin{itemize}
		\item Helps in analyte identification
		\item Chemometrics
		\item Note: Chemometrics is the use of mathematical and statistical methods to improve the understanding of chemical information and to correlate quality parameters or physical properties to analytical instrument data.
	\end{itemize}
	\item Speed
	\item Disadvantage: cost compared to single and variable wavelength detectors
\end{itemize}

\subsubsection{Refractive Index (RI) Detector}

\begin{itemize}
	\item A bulk-property detector, as monitors changes in the index of refraction of the column eluent with single or reference cells
	\item Monitors sodium D-line (589 nm) emitted for sodium lamp
	\item Advantages:
	\begin{itemize}
		\item Universal detector as it responds to most compounds as long as they have different refraction index than mobile phase
		\item Nondestructive
	\end{itemize}
	\item Disadvantages:
	\begin{itemize}
		\item Poor sensitivity ($\mu{}g$ to $ng$ in very favorable circumstances)
		\item Very temperature sensitive, needs control to \SI{0.001}{\celsius} for accurate work (i.e., sensitive to operating conditions)
		\item Not compatible with gradient elution, as changing mobile phase composition also changes refractive index and therefore baseline. Even differential instrument can’t keep up
	\end{itemize}
\end{itemize}

\subsubsection{Fluorescence Detector}

\begin{itemize}
	\item Responds to compounds that can be excited to fluoresce e.g., with a xenon lamp or more commonly with a laser
	\item A selective detector, extremely sensitive (pg to fg LODs)
	\item Very popular since many pharmaceutical products and environmental contaminants are fluorescent
	\begin{itemize}
		\item Require rigid (e.g., fused ring) system for optimal fluorescence
		\item Note that many coeluting analytes in the sample matrix aren’t fluorescent and therefore won’t interfere
		\item Precolumn or postcolumn derivatization (on-line) can introduce a fluorescent label into a compound, e.g., dansylation of \ang{1} or \ang{2} amines
		\begin{itemize}
			\item Note that this will change chromatography as a different chemical must be chromatographed!
		\end{itemize}
	\end{itemize}
\end{itemize}

\subsubsection{Electrochemical Detectors}

\begin{itemize}
	\item Useful for analytes that can be oxidized/reduced
	\item Example: amperometric thin-layer detector
	\begin{itemize}
		\item working electrode (e.g., glassy carbon) embedded in wall of flow channel and held at constant voltage ($E_{app}$)
		\item Current flows when a compound undergoes a redox reaction at the applied potential, with current $\propto$ C
		\item Applied potential to electrode is adjustable, so detector may discriminate between different electroactive species
	\end{itemize}
	\item Other detectors possible e.g., voltammetric, conductometric, coulometric
	\item pg detection limit for many analytes
	\item Mobile phase must be conductive AND isocratic, e.g., aqueous mobile phase ion chromatography or reversed-phase
\end{itemize}

\subsubsection{Evaporative Light-Scattering Detector (ELSD)}

\begin{itemize}
	\item column effluent nebulized into fine aerosol using air or \ce{N2} from nozzle
	\item Mobile phase aerosol is evaporated in a heated drift tube
	\item Fine particulates (the analytes) pass through a laser beam  scatter the laser light
	\item Scattered light measured at perpendicular angle to flow
	\item Detector response is about the same for all nonvolatile analytes
	\item Detection limits more sensitive than refractive index detector
	\item Buffers in mobile phase must be used with care, as they must volatilize to avoid major interference with signal
\end{itemize}

\subsection{Column Packings}

There are two types:

\begin{itemize}
	\item Microporous HPLC column packings
	\item Bonded phase HPLC packings
\end{itemize}

\subsubsection{Microporous HPLC Column Packings}

\begin{itemize}
	\item Microporous HPLC packings currently used for columns:
	\begin{itemize}
		\item small diameter (1.7, 1.8, 2, 3, 5, and 10 $\mu{}m$ $d_p$)
		\item fully porous
		\item silica-based
	\end{itemize}
	\item Characterized by:
	\begin{itemize}
		\item High efficiency: small diffusional distances, so both stationary and mobile phase mass-transfer terms improved
		\item High sample capacity, given high surface area since packing material is fully porous
		\item Surface is fully hydrolyzed silica (silica heated with 0.1 M \ce{HCl} for 1-2 days) with chemically reactive silanol groups
	\end{itemize}
\end{itemize}

\subsubsection{Bonded Phase HPLC Packings}

\begin{itemize}
	\item Bonded-phase packings have microporous silica support to which a liquid phase is covalently bonded
	\item No column bleed as with liquid-liquid columns
	\item Stationary phase produced by silanization
	\begin{itemize}
		\item Choice of R determines functionality of stationary phase
		\begin{itemize}
			\item e.g., R = \ce{C18H37} (bulky reagent reacting with $<=50\%$ of surface silanol groups due to steric factors)
		\end{itemize}
		\item Many polar active sites left on surface which produce tailing problems in HPLC affecting performance
		\item To reduce this effect, end-capping with less bulky reagent to react with most free silanol groups
		\item A bulky i-butyl group can be used instead of a \ce{CH3} group to increase stability (e.g., acid-catalyzed hydrolysis)
	\end{itemize}
\end{itemize}

\subsection{Types of Chromatography in HPLC}


\section{GC-MS}


\section{Important Tips}

\begin{description}
	\item[Isocratic elution] one solvent, or constant solvent mixture.
	\item[Gradient elution] continuous change of solvent composition to increase eluent strength. \\ Gradient elution in HPLC is analoguous to temperature programming in gas chromatography. \\ Increased eluent strength is required to elute more strongly retained solutes.
	\item[General elution problem] for a complex mixture, isocratic conditions can often be found to produce adequate separation of early-eluting peaks or late-eluting peaks, but not both. This problem drives us to use gradient elution.
	\item[Note:] Elution strength decreases as the solvent becomes more polar, correct???
	\item[Separation factor, $\alpha$] Also called relative retention; for two components, 1 and 2, it is the ratio of their adjusted retention times. \\ The greater the relative retention, the greater the separation between two components. \\ Relative retention is fairly independent of flow rate and can therefore be used to to help identify peaks when the flow rate changes. (show equation?)
\end{description}




\end{document}
