\documentclass[a4paper, 12pt]{article}
\usepackage[version=4]{mhchem} % Package for chemical equation typesetting
\usepackage{siunitx} % Provides the \SI{}{} and \si{} command for typesetting SI units
\usepackage{amsmath} % Required for some math elements 
\usepackage{times} % Uncomment to use the Times New Roman font
\usepackage[backend=biber, bibencoding=utf8, style=chem-acs, citestyle=chem-acs, sorting=none]{biblatex}
\setlength{\parskip}{1em}
% \addbibresource{references.bib}
\author{Robby \textsc{Renz}}
\title{\textbf{High Performace Liquid Chromatography} \\ A Summary}

\begin{document}
\maketitle
\newpage

\tableofcontents
\listoffigures
\listoftables
\newpage

\section{Introduction}
\begin{itemize}
	\item HPLC stands for "High Performance Liquid Chromatography" or "High Pressure Liquid Chromatography"
	\item \textbf{Advantages}
	\begin{itemize}
		\item Analysis of thermally unstable compounds
		\item Analysis of nonvolatile compounds
	\end{itemize}
	\item \textbf{Major requirement of LC}
	\begin{itemize}
		\item Solute solubility in mobile phase
		\item This is in contrast to GC which require solute volatility
	\end{itemize}
\end{itemize}


\section{Scope of HPLC}


\section{Column Efficiency in HPLC}


\section{Pumps}


\section{Elution Techniques}


\section{Injectors}


\section{Columns}


\section{Detectors}


\section{Types of Chromatography in HPLC}






\end{document}

