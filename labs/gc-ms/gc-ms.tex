%----------------------------------------------------------------------------------------
%	PACKAGES AND DOCUMENT CONFIGURATIONS
%----------------------------------------------------------------------------------------

\documentclass[a4paper, 12pt]{article}
\usepackage[version=4]{mhchem} % Package for chemical equation typesetting
\usepackage{siunitx} % Provides the \SI{}{} and \si{} command for typesetting SI units
\usepackage{amsmath} % Required for some math elements 
\usepackage{times} % Uncomment to use the Times New Roman font
\usepackage[backend=biber, bibencoding=utf-8, style=chem-acs, citestyle=chem-acs, sorting=none]{biblatex}
\setlength{\parskip}{1em}
\addbibresource{references.bib}
\usepackage{geometry}
\geometry{margin=1in}
\usepackage{textcomp}
% \usepackage{graphicx}
% \graphicspath{ {./images/} }

%----------------------------------------------------------------------------------------
%	DOCUMENT INFORMATION
%----------------------------------------------------------------------------------------

\title{Analysis and Identification of a Thingy using GC-MS \\ CHEM 4303 \\ Analytical Separations} % Title

\author{Robby \textsc{Renz}} % Author name

\date{\today} % Date for the report

\begin{document}

\maketitle % Insert the title, author and date

\begin{center}
\begin{tabular}{l r}
Date Performed: & November 13, 2018 \\ % Date the experiment was performed
Date Completed: & November 27, 2018 \\
Partner: & Jaya Roe \\ % Partner name
Lab Instructor: & Kevin Stroski % Instructor/supervisor
\end{tabular}
\end{center}

%----------------------------------------------------------------------------------------
%	SECTION 1
%----------------------------------------------------------------------------------------

\begin{abstract}
	Gas chromatography with a mass spectrometer was utilized for analysis.
\end{abstract}
\newpage

%----------------------------------------------------------------------------------------
%	SECTION 2
%----------------------------------------------------------------------------------------

\section{Introduction}
Gas chromatography (GC) is a separation technique that analyzes volatile compounds \cite{vitha_chromatography:_2017}. Consequently, this analysis can lead a lot of useful situations such as the determination of the purity of a compound or even detecting explosives \cite{vitha_chromatography:_2017}. In GC, the analyte is volatilized and carried through the column by the mobile phase, often called the carrier gas \cite{harris}. This carrier gas can either be \ce{He} or \ce{H2} \cite{harris}. These gases are often chosen as the carrier gas as they are chemically inert and would therefore not react with the analytes \cite{vitha_chromatography:_2017}.

Polybrominated diphenyl ethers (PBDEs) are a class of halogenated compounds that are commonly used as flame retardants \cite{bjorklund_mass_2003}. These compounds are an environmental health hazard as they have the potential to accumulate in the food chain \cite{thomsen_comparing_2002}. In addition, BDE-47, a PBDE congener, has been found to cause neurotoxic effects in adults \cite{thomsen_comparing_2002}. Commonly used detection techniques for PBDEs are high-resolution mass spectrometry and low-resolution mass spectrometry (LRMS) \cite{bjorklund_mass_2003}. LRMS is commonly done with selected ion monitoring (SIM) \cite{bjorklund_mass_2003}. SIM increases the selectivity of mass spectrometry for analytes and reduces its response to everything else \cite{harris}.

The main objective of this experiment is to make a method that uses a SIM for quantitative analysis of PBDEs by GC-EI-LRMS.

%----------------------------------------------------------------------------------------
%	SECTION 3
%----------------------------------------------------------------------------------------

\section{Chemicals, Methods and Instrumentation}

\subsection{Chemicals}
In the first week, BDE-47, PBB-77 and `2-\ce{HCH}, all of each were at a concentration of \SI{50}{\mu{}g/mL} in isooctane, were used as standard solutions. In the second week, fish oil (Exact Norwegian Cod Liver Oil), dichloromethane (emd, Lot 5Q160, CAS: 75-09-2) and PBB-77, at a concentration of \SI{10}{\mu{}g/mL} in isooctane, were used as chemicals. Finally, in the last week, BDE-47, PBB-77 and `2-\ce{HCH}, each at a concentration of \SI{10}{\mu{}g/mL} in isooctane, were used. Throughout the entirety of the experiment, hexane (Caledon Laboratory Chemicals, CAS no. 110-54-3, LOT: 89001) and isooctane (OmniSolv, CAS: 540-84-1, LOT: 52054) were used.

\subsection{Instrumentation}
The separation and analysis of the entire experiment was performed on an Agilent 7890A GC, coupled with a 5975C inert XL EI/CI MSD with a triple axis detector. The dimensions of the column used was $30m \times 0.250mm \times 0.25\mu{}m$, by Agilent Technologies. The stationary phase was (5\%-Phenyl)-methylpolysiloxane. Each analysis was performed with the injection mode at splitless, with \ce{He} as the carrier gas, and the flow rate was set at \SI{1}{mL/min}. Do I need to state the pressure and volume and splitless?

\subsection{Methods}

%----------------------------------------------------------------------------------------
%	SECTION 4
%----------------------------------------------------------------------------------------

\section{Results and Discussion}

\subsection{Results}
Table \ref{tab-para} shows some cool stuff, and table \ref{tab-mass} shows the molecular weight of BDE-47, PBB-77 and `2-\ce{HCH}. Table \ref{tab-fish} shows the mass of fish oil that was weighed out for the experiment.

\begin{table}[h!]
	\centering
	\caption{Selected Ion Monitoring Parameters}
	\hfill \\
	\begin{tabular}{|c|c|c|}
		\hline
		Compound & Ions monitored (m/z) & Time window (min) \\
		\hline
		`2-\ce{HCH} & 181, 219 & 0.5 - 19 \\
		\hline
		BDE-47 & 326, 486 & 19 - 22.5 \\
		\hline
		PBB-77 & 470, 310 & 22.5 - 24 \\
		\hline
	\end{tabular}
	\label{tab-para}
\end{table}

\begin{table}[h!]
	\centering
	\caption{Molecular weight of BDE-47, PBB-77 and `2-HCH}
	\hfill \\
	\begin{tabular}{|c|c|}
		\hline
		Compound & Molecular weight (\si{\amu}) \\
		\hline
		BDE-47 & 486 \\
		\hline
		PBB-77 & 470 \\
		\hline
		`2-\ce{HCH} & 291 \\
		\hline
	\end{tabular}
	\label{tab-mass}
\end{table}

\begin{table}[h!]
	\centering
	\caption{Mass of fish oil weighed out}
	\hfill \\
	\begin{tabular}{|c|c|}
		\hline
		Sample number & Mass (\si{\gram}) \\
		\hline
		1 & 1.1234 \\
		\hline
		2 & 0.9062 \\
		\hline
	\end{tabular}
	\label{tab-fish}
\end{table}


\subsection{Discussion}


%----------------------------------------------------------------------------------------
%	SECTION 5 (BIBLIOGRAPHY)
%----------------------------------------------------------------------------------------

\section{References}
\printbibliography

%----------------------------------------------------------------------------------------
%	SECTION 6
%----------------------------------------------------------------------------------------

\section{Appendix}

\subsection{Calculations}

\begin{description}
	\item[Calculating the capacity factor of chlorobenzene for figure 2 of the chromatogram] \hfill \\
		\begin{equation} \label{equ-capacity-factor}
			\begin{split}
				K & = \frac{t_r - t_m}{t_m} \\
				K & = \frac{2.343 - 1.438}{1.438} \\
				K & = \frac{0.905}{1.438} \\
				K & = 0.629346 \approx 0.629
			\end{split}
		\end{equation}
		Equation \ref{equ-capacity-factor} was taken from \cite{harris}.
\end{description}

\subsection{Chromatograms}
There are ? sheets of GC-MS chromatograms.

%----------------------------------------------------------------------------------------


\end{document}
