\documentclass[a4paper, 12pt]{article}
\usepackage{amsmath}
\usepackage[version=4]{mhchem}
\usepackage[backend=biber, bibencoding=utf8, style=chem-acs, citestyle=chem-acs, sorting=none]{biblatex}
\setlength{\parskip}{1em}
\addbibresource{references.bib}
\usepackage{geometry}
\geometry{margin=1in}
\title{Detection and Quantitation of PBDEs by GC-EI-LRMS}
\author{Robby \textsc{Renz}}

\begin{document}
\maketitle

\section{Introduction}
Introduction is here. Talk about the instrument in general (GC-MS), then talk about GC-EI-LRMS, maybe talk about the detector why use it etc.

The main objective of this experiment is to\dots

\section{Chemicals and their Hazards}

\begin{description}
	\item[BDE-47] full form: 2,2',4,4'-tetrabromodiphenyl ether; may be harmful if inhaled, swallowed, may cause skin and eye irritation.
	\item[PBB-77] full form: 3,3',4,4'-tetrabromobiphenyl; 
	\item[`2-HCH] also known as lindane; may cause acute toxicity, health hazard, hazardous to the aquatic environment.
\end{description}

\printbibliography

\end{document}

