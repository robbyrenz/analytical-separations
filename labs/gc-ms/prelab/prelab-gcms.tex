\documentclass[a4paper, 12pt]{article}
\usepackage{amsmath}
\usepackage[version=4]{mhchem}
\usepackage[backend=biber, bibencoding=utf8, style=chem-acs, citestyle=chem-acs, sorting=none]{biblatex}
\setlength{\parskip}{1em}
\addbibresource{references.bib}
\usepackage{geometry}
\geometry{margin=1in}
\title{Detection and Quantitation of PBDEs by GC-EI-LRMS}
\author{Robby \textsc{Renz}}

\begin{document}
\maketitle

\section{Introduction}
Gas chromatography (GC) is a separation technique that analyzes volatile compounds \cite{vitha_chromatography:_2017}. Consequently, this analysis can lead a lot of useful situations such as the determination of the purity of a compound or even detecting explosives \cite{vitha_chromatography:_2017}. In GC, the analyte is volatilized and carried through the column by the mobile phase, often called the carrier gas \cite{harris}. This carrier gas can either be \ce{He} or \ce{H2} \cite{harris}. These gases are often chosen as the carrier gas as they are chemically inert and would therefore not react with the analytes \cite{vitha_chromatography:_2017}.

Polybrominated diphenyl ethers (PBDEs) are a class of halogenated compounds that are commonly used as flame retardants \cite{bjorklund_mass_2003}. These compounds are an environmental health hazard as they have the potential to accumulate in the food chain \cite{thomsen_comparing_2002}. In addition, BDE-47, a PBDE congener, has been found to cause neurotoxic effects in adults \cite{thomsen_comparing_2002}. Commonly used detection techniques for PBDEs are high-resolution mass spectrometry and low-resolution mass spectrometry (LRMS) \cite{bjorklund_mass_2003}. LRMS is commonly done with selected ion monitoring (SIM) \cite{bjorklund_mass_2003}. SIM increases the selectivity of mass spectrometry for analytes and reduces its response to everything else \cite{harris}.

The main objective of this experiment is to make a method that uses a SIM for quantitative analysis of PBDEs by GC-EI-LRMS.

\section{Chemicals and their Hazards}

\begin{description}
	\item[BDE-47] full form: 2,2',4,4'-tetrabromodiphenyl ether; may be harmful if inhaled, swallowed, may cause skin and eye irritation.
	\item[PBB-77] full form: 3,3',4,4'-tetrabromobiphenyl; flammable, acute toxicity, hazardous to the aquatic environment, may cause skin irritation.
	\item[`2-HCH] also known as lindane; may cause acute toxicity, health hazard, hazardous to the aquatic environment.
\end{description}

\printbibliography

\end{document}

