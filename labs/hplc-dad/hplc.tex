%----------------------------------------------------------------------------------------
%	PACKAGES AND DOCUMENT CONFIGURATIONS
%----------------------------------------------------------------------------------------

\documentclass[a4paper, 12pt]{article}
\usepackage[version=4]{mhchem} % Package for chemical equation typesetting
\usepackage{siunitx} % Provides the \SI{}{} and \si{} command for typesetting SI units
\usepackage{amsmath} % Required for some math elements 
\usepackage{times} % Uncomment to use the Times New Roman font
\usepackage[backend=biber, bibencoding=utf-8, style=chem-acs, citestyle=chem-acs, sorting=none]{biblatex}
\setlength{\parskip}{1em}
\addbibresource{references.bib}
\usepackage{geometry}
\geometry{margin=1in}

%----------------------------------------------------------------------------------------
%	DOCUMENT INFORMATION
%----------------------------------------------------------------------------------------

\title{Separation and Analysis of Pharmaceuticals using RP-HPLC \\ CHEM 4303 \\ Analytical Separations} % Title

\author{Robby \textsc{Renz}} % Author name

\date{\today} % Date for the report

\begin{document}

\maketitle % Insert the title, author and date

\begin{center}
\begin{tabular}{l r}
Date Performed: & October 16, 2018 \\ % Date the experiment was performed
Date Completed: & October 30, 2018 \\
Partner: & Jaya Roe \\ % Partner name
Lab Instructor: & Kevin Stroski % Instructor/supervisor
\end{tabular}
\end{center}

%----------------------------------------------------------------------------------------
%	SECTION 1
%----------------------------------------------------------------------------------------

\begin{abstract}
	High-performance liquid chromatography is a separation technique blah blah blah \dots
\end{abstract}
\newpage

%----------------------------------------------------------------------------------------
%	SECTION 2
%----------------------------------------------------------------------------------------

\section{Introduction}
High performance liquid chromatography (HPLC) is a type of chromatographic technique in which this instrumentation uses high pressure in order to push the solvent through the column \cite{harris}. In an HPLC, the stationary phase is usually inorganic particles that are made up of silica that are porous, and the type of chromatographic techniques that utilizes this type of stationary phase are normal phase (NP), ion pair, and reversed phase (RP) \cite{mold}. While NP chromatography is usually utilized in the purification of organic molecules, RP is used to purify proteins that are dissolved in organic solvents and aqueous buffers and the adsorption between the stationary phase and the solute becomes unhinged with the increase in concentration of the aforementioned solvent \cite{prep_hplc}. In RP, the solvent is less polar than the stationary phase but has the higher eluent strength \cite{harris}. 

Acetaminophen (APAP), an antipyretic and a famous analgesic for humans, is known to cause poisons in cats and dogs \cite{dogs-cats}. And para-Aminophenol (PAP), an APAP metabolite, is a known nephrotoxicant for rats \cite{rats}. Thus, there is a need to separate as well as analyze these pharmaceutical components, and that is the main purpose of this experiment. And this is done with an RP-HPLC.

The main objectives of this experiment is to \ldots{}

%----------------------------------------------------------------------------------------
%	SECTION 3
%----------------------------------------------------------------------------------------

\section{Methods and Instrumentation}

\subsection{Chemicals}
Toluene (HPLC grade chemical, LOT: 591103-A6, CAS: 108-88-3), naphthalene (Fisher Scientific, LOT: 895861, CAS: 91-20-3), and a mixture of PAH (PAH mix 1, LOT w00382; PAH mix 2, CD-1661; Ultra EPA 2138N-1, EOA 2139N-1, ACN) were used in this experiment.

The safety information for the aforementioned chemicals: (this might not be needed...)
\begin{description}
	\item[Toluene] flammable, potential acute and chronic health effects.
	\item[Naphthalene] flammable, toxic and carcinogenic.
	\item[PAH mixture] flammable, can cause death, health hazard, can cause damage to the aquatic environment.
\end{description}

\subsection{Instrumentation}
The separation and analysis of the pharmaceutical mixture was performed on an Agilent 6890N Network GC System (Agilent Technologies), equipped with an dash detector. The separation was performed on an Agilent(?) 122-5032 J\&W DB-5 capillary column, \SI{30}{m} length $\times$ \SI{0.25}{mm} i.d. $\times$ \SI{0.25}{\mu{}m} thickness, \SI{7}{inch} cage. Its stationary phase was 5$\%$ Phenyl/95$\%$ methylpolysiloxane (DB-5). In addition, each analysis was conducted with the injection mode at splitless and the carrier gas in the GC-FID was \ce{He}, with a flow rate of \SI{1}{\frac{mL}{min}}. The injector port temperature was set to \SI{260}{\degreeCelsius} and for the analyses of toluene and naphthalene, the detector temperature was set to \SI{280}{\degreeCelsius}, and it was increased to \SI{300}{\degreeCelsius} for the analyses of the pharmaceutical mixture and the standards.

\subsection{Methods}


%----------------------------------------------------------------------------------------
%	SECTION 4
%----------------------------------------------------------------------------------------

\section{Results and Discussion}

\subsection{Results}
Write your results here\ldots


\subsection{Discussion}
Write your discussion here\dots

The reason for using internal standards and not standard addition for calculating the concentration of the unknown analyte is that internal standards are preferred when give an accurate answer when 

In conclusion\dots

%----------------------------------------------------------------------------------------
%	SECTION 5 (BIBLIOGRAPHY)
%----------------------------------------------------------------------------------------

\section{References}
\printbibliography

%----------------------------------------------------------------------------------------
%	SECTION 6
%----------------------------------------------------------------------------------------

\section{Appendix}

\subsection{Calculations}

\begin{description}
	\item[Calculating the response factor] \hfill \\
		\begin{equation} \label{response-factor}
		\begin{split}
			\frac{\textit{Area of Analyte Signal}}{\textit{Concentration of Analyte}} & = F\Bigg(\frac{\textit{Area of Standard Signal}}{\textit{Concentration of Standard}}\Bigg) \\
			\frac{A_X}{[X]} & = F\Bigg(\frac{A_S}{[S]}\Bigg)
		\end{split}
		\end{equation}
			Equation \ref{response-factor} was taken from \cite{harris}, where $[X]$ and $[S]$ represent the concentrations of analyte and of the standard.
\end{description}

\subsection{Chromatograms}
There are 50(?) pages of printouts that were collected during this experiment, 6 of which are GC-FID chromatograms.(?)


%----------------------------------------------------------------------------------------


\end{document}
