%----------------------------------------------------------------------------------------
%	PACKAGES AND DOCUMENT CONFIGURATIONS
%----------------------------------------------------------------------------------------

\documentclass[a4paper, 12pt]{article}
\usepackage[version=4]{mhchem} % Package for chemical equation typesetting
\usepackage{siunitx} % Provides the \SI{}{} and \si{} command for typesetting SI units
\usepackage{amsmath} % Required for some math elements 
\usepackage{times} % Uncomment to use the Times New Roman font
\usepackage[backend=biber, bibencoding=utf-8, style=chem-acs, citestyle=chem-acs, sorting=none]{biblatex}
\setlength{\parskip}{1em}
\addbibresource{references.bib}
\usepackage{geometry}
\geometry{margin=1in}
\usepackage{textcomp}

%----------------------------------------------------------------------------------------
%	DOCUMENT INFORMATION
%----------------------------------------------------------------------------------------

\title{Separation and Analysis of An Unknown Pharmaceutical mixture using RP-HPLC \\ CHEM 4303 \\ Analytical Separations} % Title

\author{Robby \textsc{Renz}} % Author name

\date{\today} % Date for the report

\begin{document}

\maketitle % Insert the title, author and date

\begin{center}
\begin{tabular}{l r}
Date Performed: & October 16, 2018 \\ % Date the experiment was performed
Date Completed: & October 30, 2018 \\
Partner: & Jaya Roe \\ % Partner name
Lab Instructor: & Kevin Stroski % Instructor/supervisor
\end{tabular}
\end{center}

%----------------------------------------------------------------------------------------
%	SECTION 1
%----------------------------------------------------------------------------------------

\begin{abstract}
	High-performance liquid chromatography utilizing the reverse-phase\dots{}is a separation technique blah blah blah \dots
\end{abstract}
\newpage

%----------------------------------------------------------------------------------------
%	SECTION 2
%----------------------------------------------------------------------------------------

\section{Introduction}
High performance liquid chromatography (HPLC) is a type of chromatographic technique in which this instrumentation uses high pressure in order to push the solvent through the column \cite{harris}. In an HPLC, the stationary phase is usually inorganic particles that are made up of silica that are porous, and the type of chromatographic techniques that utilizes this type of stationary phase are normal phase (NP), ion pair, and reversed phase (RP) \cite{mold}. While NP chromatography is usually utilized in the purification of organic molecules, RP is used to purify proteins that are dissolved in organic solvents and aqueous buffers and the adsorption between the stationary phase and the solute becomes unhinged with the increase in concentration of the aforementioned solvent \cite{prep_hplc}. In RP, the solvent is less polar than the stationary phase but has the higher eluent strength \cite{harris}. 

Acetaminophen (APAP), an antipyretic and a famous analgesic for humans, is known to cause poisons in cats and dogs \cite{dogs-cats}. And para-Aminophenol (PAP), an APAP metabolite, is a known nephrotoxicant for rats \cite{rats}. Thus, there is a need to separate as well as analyze these pharmaceutical components, and that is the main purpose of this experiment. And this is done with an RP-HPLC.

The main objectives of this experiment is to \ldots{}

%----------------------------------------------------------------------------------------
%	SECTION 3
%----------------------------------------------------------------------------------------

\section{Chemicals, Methods and Instrumentation}

\subsection{Chemicals}
Toluene (HPLC grade chemical, LOT: 591103-A6, CAS: 108-88-3), naphthalene (Fisher Scientific, LOT: 895861, CAS: 91-20-3), and a mixture of PAH (PAH mix 1, LOT w00382; PAH mix 2, CD-1661; Ultra EPA 2138N-1, EOA 2139N-1, ACN) were used in this experiment.

\subsection{Instrumentation}
The separation and analysis of the unknown pharmaceutical mixtures, and the standards, were performed on an Agilent 1100 Series, with a 1260 Infinity degasser (both by Agilent Technologies), fitted with a diode array detector (DAD). The type of column used was a Symmetry\textregistered{} \ce{C18} (particle size of \SI{5}{\mu{}m}, diameter of \SI{4.6}{mm}, and a length of \SI{150}{mm}). The flow rate for each analysis was kept at a constant value of \SI{1}{mL/min}, and the injection volume was \SI{5}{\mu{}L}, for each analysis.

\subsection{Methods}
Write your methods here \dots

%----------------------------------------------------------------------------------------
%	SECTION 4
%----------------------------------------------------------------------------------------

\section{Results and Discussion}

\subsection{Results}
As stated earlier, (maybe?), in the first week of the experiment, the compositions of the mobile phase, methanol, was being varied with HPLC-grade water in order to notice the differences in separations in reverse-phase HPLC. The corresponding chromoatograms are given in the appendix (figures 1 to 4), and table \ref{tab-capacity-factors} summarizes the aforementioned chromotagrams.

\begin{table}[h!]
	\centering
	\begin{tabular}{|c|c|c|}
		\hline
		Methanol/\% & k' & log k' \\
		\hline
		100 & & \\
		\hline
		90 & & \\
		\hline
		80 & & \\
		\hline
		70 & & \\
		\hline
	\end{tabular}
	\caption{Data that shows how the capacity factors (k) differ with the change in composition in methanol}
	\label{tab-capacity-factors}
\end{table}

In addition, figure \ref{fig-capacity-factors} illustrates the log of the capacity factors as a function of the polarity of methanol.

Table \ref{tab-selectivity} now shows how the selectivity factor ($\alpha$) varies with the 

\begin{table}[h!]
	\centering
	\begin{tabular}{|c|c|c|}
		\hline
		Methanol/\% & k' & log k' \\
		\hline
		100 & & \\
		\hline
		90 & & \\
		\hline
		80 & & \\
		\hline
		70 & & \\
		\hline
	\end{tabular}
	\caption{Data that shows how the selectivity factors ($\alpha$) change with the polarity of methanol}
	\label{tab-selectivity}
\end{table}

Figures 5, 6 and 7 show the chromatograms of a chlorobenzene, nitrobenzene and benzaldehyde standards, respectively. In addition, figure 8 illustrates the chromatogram of the unknown pharmaceutical mixture. While figures 9, 10, 11 and 12 shows the chromatograms of the caffeine, phenacetin, salicylamide and the acetominophen standard, respectively.

Finally, in the quantitative analysis of the experiment, figures 13, 14, 15, 16 and 17 represent the chromatograms \dots{}. Figures 18 and 19 show the chromatograms of the analysis of the aforementioned pharmaceutical unknown. Finally, figures 20, 21 and 22 represents the chromatograms of the blank that was run (50\% methanol).

The pressure of the column(?) for each analysis is shown in table \ref{tab-pressure}. Note that the figure headings correspond to the analysis of the chromatograms in the appendix.

\begin{table}[h!]
	\centering
	\begin{tabular}{|c|c|}
		\hline
		Figure number & Pressure/bar\\
		\hline
		Figure 1  & 72 \\ 
		\hline
		Figure 2 & 98 \\
		\hline
		Figure 3 & 122 \\
		\hline
		Figure 4 & 143 \\
		\hline
		Figure 5 & 143 \\
		\hline
		Figure 6 & 143 \\
		\hline
		Figure 7 & 143 \\
		\hline
		Figure 8 & 178 \\
		\hline
		Figure 9 & 178 \\
		\hline
		Figure 10 & 178 \\
		\hline
		Figure 11 & 178 \\
		\hline
		Figure 12 & 178 \\
		\hline
		Figure 13 & \\
		\hline
		Figure 14 & \\
		\hline
		Figure 15 & \\
		\hline
		Figure 16 & \\
		\hline
		Figure 17 & \\
		\hline
		Figure 18 & \\
		\hline
		Figure 19 & \\
		\hline
		Figure 20 & \\
		\hline
		Figure 21 & \\
		\hline
		Figure 22 &  \\ 
		\hline
	\end{tabular}
	\caption{Table that tabulates how the selectivity factors ($\alpha$) change with the polarity of methanol}
	\label{tab-pressure}
\end{table}

\textbf{Now draw a table for the data of the calibration curve, and include a figure.}

\subsection{Discussion}
Give a reason as to why reverse-phase is used as a separation technique and not normal phase.

Maybe state why the pressure increases as the polarity of the mobile phase increases. Hint: its because of the hydrogen bonding of the water molecules, hence a larger pressure is needed to push it through the column.

In the quantitative identification of the pharmaceutical unknown mixture, the initial concentration of the mobile phase was chosen as 50\% methanol, and based on the chromatogram received, this composition was considered to be desirable for the analysis of the aforementioned unknown mixture. This was based on the resolution of the peaks and the retention time for each peaks.

The reason for using internal standards and not standard addition for calculating the concentration of the unknown analyte is that internal standards are preferred when give an accurate answer when 

In conclusion, reversed-phase HPLC gives a \dots

%----------------------------------------------------------------------------------------
%	SECTION 5 (BIBLIOGRAPHY)
%----------------------------------------------------------------------------------------

\section{References}
\printbibliography

%----------------------------------------------------------------------------------------
%	SECTION 6
%----------------------------------------------------------------------------------------

\section{Appendix}

\subsection{Calculations}

\begin{description}
	\item[Calculating the response factor] \hfill \\
		\begin{equation} \label{response-factor}
		\begin{split}
			\frac{\textit{Area of Analyte Signal}}{\textit{Concentration of Analyte}} & = F\Bigg(\frac{\textit{Area of Standard Signal}}{\textit{Concentration of Standard}}\Bigg) \\
			\frac{A_X}{[X]} & = F\Bigg(\frac{A_S}{[S]}\Bigg)
		\end{split}
		\end{equation}
			Equation \ref{response-factor} was taken from \cite{harris}, where $[X]$ and $[S]$ represent the concentrations of analyte and of the standard.

	\item[$\%$RSD] \hfill \\
\end{description}

\subsection{Chromatograms}
There are 22 sheets of HPLC chromatograms.

%----------------------------------------------------------------------------------------


\end{document}
