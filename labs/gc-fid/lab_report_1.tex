%----------------------------------------------------------------------------------------
%	PACKAGES AND DOCUMENT CONFIGURATIONS
%----------------------------------------------------------------------------------------

\documentclass[a4paper, 12pt]{article}
\usepackage[version=4]{mhchem} % Package for chemical equation typesetting
\usepackage{siunitx} % Provides the \SI{}{} and \si{} command for typesetting SI units
\usepackage{amsmath} % Required for some math elements 
\usepackage{times} % Uncomment to use the Times New Roman font
\usepackage[backend=biber, style=draft, citestyle=chem-acs, sorting=none]{biblatex}
\addbibresource{sample.bib}

%----------------------------------------------------------------------------------------
%	DOCUMENT INFORMATION
%----------------------------------------------------------------------------------------

\title{Determination of Optimized Parameters for the Separation of PAHs using Gas Chromatography \\ CHEM 4303 \\ Analytical Separations} % Title

\author{Robby \textsc{Renz}} % Author name

\date{\today} % Date for the report

\begin{document}

\maketitle % Insert the title, author and date

\begin{center}
\begin{tabular}{l r}
Date Performed: & September 11, 2018 \\ % Date the experiment was performed
Date Completed: & October 2, 2018 \\
Partner: & Jaya Roe \\ % Partner name
Lab Instructor: & Kevin Stroski % Instructor/supervisor
\end{tabular}
\end{center}

%----------------------------------------------------------------------------------------
%	SECTION 1
%----------------------------------------------------------------------------------------

\begin{abstract}
	The main objective of this experiment was to separate polycyclic aromatic hydrocarbons (PAHs). This was done with the help of a gas chromatography equipped with a flame ionization detector \cite{Smith:2012qr}. 
\end{abstract}

\newpage

%----------------------------------------------------------------------------------------
%	SECTION 2
%----------------------------------------------------------------------------------------

\section{Introduction}
Separation is important and stuff.

%----------------------------------------------------------------------------------------
%	SECTION 3
%----------------------------------------------------------------------------------------

\section{Objectives}

\begin{description}
	\item[First Objective] \hfill \\
	Making the solution.
	\item[Second Objective] \hfill \\
	Running the GC-FID.
\end{description}
 
%----------------------------------------------------------------------------------------
%	SECTION 4
%----------------------------------------------------------------------------------------

\section{Methods and Instrumentation}
The gas chromatography, equipped with the flame ionisation detector, was the instrument used to separate the PAH's.

%----------------------------------------------------------------------------------------
%	SECTION 5
%----------------------------------------------------------------------------------------

\section{Results and Discussion}

%----------------------------------------------------------------------------------------
%	SECTION 6 (BIBLIOGRAPHY)
%----------------------------------------------------------------------------------------

\section{References}
\printbibliography

%----------------------------------------------------------------------------------------
%	SECTION 7
%----------------------------------------------------------------------------------------

\section{Appendix}

%----------------------------------------------------------------------------------------


\end{document}
